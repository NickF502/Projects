%Name: Nicholas Farkash
%Purpose: To gain experience with producing a document using LaTeX
%Date: November 29th, 2021
%Due: December 6th 2021

\documentclass[12pt]{article}
\usepackage{epsfig}
\begin {document}

\begin{center}
  Nicholas Farkash - Assignment 15
\end{center}

\section{Introduction}
In Assignment 10, we used Poisson's Equation, a partial different equation, for
a 2-D steady state electromagnetics problem to determine the electric potential
of a system at a particular point. In this document, the process of determining
this value will be briefly explored.


\section{Equation}
Poisson's Equation is a partial differential equation. When applied to our
two-dimensional electomagnetics problem, it takes the form:

\begin{equation} \frac{\partial^2 U}{\partial x^2} + \frac{\partial^2 U}{\partial
                 y^2} = 4 \pi q \label{Poisson}
\end{equation}

where $U$ is the electrical potential and $q$ is the electrical charge density.

In word form, this equation says that the second derivative of the electric
potential with respect to the x direction plus the second derivative of the
electric potential with respect to the y direction is equal to 4$\pi$ times
the electrical charge density.

\subsection{Finite-Difference Form}
If we divide the box we are observing into square-shaped cells of sides $h$, we
can refer to each individual cell by its location $U_{i,j}$, which we can then
use to loop over in our calculations. By approximating the first derivatives of
the electric potential at the edge of each cell, and then the second derivatives
of the electric potential at the center of each cell, we can calculate the
charge density of any particular cell.

The first derivatives are approximated as follows:
\begin{equation}
  \left(\frac{\partial U}{\partial x}\right)_{i,j+\frac{1}{2}} \approx
  \frac{U_{i,j+1}-U_{i,j}}{h}
\hspace{24pt}
  \left(\frac{\partial U}{\partial y}\right)_{i+\frac{1}{2},j} \approx
  \frac{U_{i+1,j}-U_{i,j}}{h}
\end{equation} 

\vspace{34pt}

The second derivatives are approximated as follows:

\begin{equation}
  \left(\frac{\partial^2 U}{\partial x^2}\right)_{i,j} \approx \frac{\left(
  \frac{\partial U}{\partial x}\right)_{i,j+\frac{1}{2}} - \left(\frac{\partial
  U}{\partial x}\right)_{i,j-\frac{1}{2}}}{h} = \frac{U_{i,j+1}-2U_{i,j}+
  U_{i,j-1}}{h^2} \label{secondx}
\end{equation}

\begin{equation}
  \left(\frac{\partial^2 U}{\partial y^2}\right)_{i,j} \approx \frac{\left(
  \frac{\partial U}{\partial y}\right)_{i+\frac{1}{2},j} - \left(\frac{\partial
  U}{\partial y}\right)_{i-\frac{1}{2},j}}{h} = \frac{U_{i+1,j}-2U_{i,j}+
  U_{i-1,j}}{h^2} \label{secondy}
\end{equation}


We can insert equations (\ref{secondx}) and (\ref{secondy}) into equation
(\ref{Poisson}) and produce the following equation:

\begin{equation}
  \frac{U_{i+1,j} + U_{i-1,j} + U_{i,j+1} + U_{i,j-1} - 4U_{i,j}}{h^2}=4
  \pi q_{i,j} \label{FinDiffForm}
\end{equation}

Solving equation (\ref{FinDiffForm}) for $U_{i,j}$ yields:

\begin{equation}
  \frac{U_{i+1,j} + U_{i-1,j} + U_{i,j+1} + U_{i,j-1} - 4 \pi h^2 q_{i,j}}
  {4}=U_{i,j} \label{SolvedForUij}
\end{equation}

This formula takes the sum of the values of the surrounding cells, subtracts
the value $4\pi h^2 q_{i,j}$, and divides this by 4 to approximate the new value.

\section{Procedure}
Using equation (\ref{SolvedForUij}), we can write a program to calculate the
electric potential. The iterative process used is called the relaxation method,
and the procedure for this method is shown below:

Begin with equation (\ref{SolvedForUij}):

\begin{enumerate}
\item Make an initial guess for the solution at each cell. For simplicity, make
      this guess 0 for all cells.
\item Use these initial guesses on the right side of equation
      (\ref{SolvedForUij}).
\item Solve for $U_{i,j}$. This is the new estimate of the solution.
\item Use the new estimate on the right side of equation (\ref{SolvedForUij}).
\item Repeat this process until the values stop changing (or are changing so
      slightly that they are negligible). This means the solution has converged
      to the answer.
\end{enumerate}

This method works well for the interior cells of the box, but what about the
cells on the boundary? When using the relaxation method on these cells, there
may not be a cell adjacent to the box whose value can be used during the
calculation. To avoid this issue, we create a bigger box around the first
box that is one cell larger in each direction. Since these new cells are
outside of the box containing the charge, they will all always have the value 0.
This gives each cell inside the box all four adjacent cells necessary to carry
out the relaxation method's calculations.

\section{Results}
For Assignment 10, we were asked to find the electric potential in the cell
(25,50). The value at this particular cell was 6.18657880E-02 CGS units.

Below is an image of a heat map, which depicts the converged solutions of
Assignment 10.

\begin{figure}[ht]
\epsfig{file=Farkash_heatmap.eps}
\caption{A heat map of the results}
\end{figure}

\end {document}